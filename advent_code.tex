\documentclass{article}

\usepackage{fontspec}
\usepackage{unicode-math}

\usepackage{fancyvrb}

\usepackage{hyperref}

\setmainfont{TeX Gyre Pagella}
\setmathfont{TeX Gyre Pagella Math}

\ExplSyntaxOn

% These are our output stack variables
\tl_new:N \g__aoc_output_tl
\int_new:N \g__aoc_output_int
\seq_new:N \g__aoc_output_seq
\bool_new:N \g__aoc_output_bool
\prop_new:N \g__aoc_output_prop

% To avoid creating vast numbers of variables, we provide ourselves with a few that we reuse frequently.
% For that reason, most of them don't have very exciting names.

% These are general purpose variables.
\tl_new:N \l__aoc_tmpa_tl
\tl_new:N \l__aoc_tmpb_tl
\tl_new:N \l__aoc_tmpc_tl
\tl_new:N \l__aoc_tmpd_tl
\tl_new:N \l__aoc_tmpe_tl
\tl_new:N \l__aoc_tmpf_tl
\tl_new:N \l__aoc_tmpg_tl
\tl_new:N \l__aoc_tmph_tl
\tl_new:N \l__aoc_tmpi_tl

\str_new:N \l__aoc_tmpa_str
\str_new:N \l__aoc_tmpb_str
\str_new:N \l__aoc_tmpc_str

\seq_new:N \l__aoc_tmpa_seq
\seq_new:N \l__aoc_tmpb_seq
\seq_new:N \l__aoc_tmpc_seq

\dim_new:N \l__aoc_tmpa_dim
\dim_new:N \l__aoc_tmpb_dim

\fp_new:N \l__aoc_tmpa_fp
\fp_new:N \l__aoc_tmpb_fp
\fp_new:N \l__aoc_tmpc_fp
\fp_new:N \l__aoc_tmpd_fp
\fp_new:N \l__aoc_tmpe_fp
\fp_new:N \l__aoc_tmpf_fp

\int_new:N \l__aoc_tmpa_int
\int_new:N \l__aoc_tmpb_int
\int_new:N \l__aoc_tmpc_int
\int_new:N \l__aoc_tmpd_int
\int_new:N \l__aoc_tmpe_int
\int_new:N \l__aoc_tmpf_int
\int_new:N \l__aoc_tmpg_int

\bool_new:N \l__aoc_tmpa_bool
\bool_new:N \l__aoc_tmpb_bool

\ior_new:N \l__aoc_tmpa_ior
\iow_new:N \l__aoc_tmpa_iow

\prop_new:N \l__aoc_tmpa_prop
\prop_new:N \l__aoc_tmpb_prop
\prop_new:N \l__aoc_tmpc_prop

\msg_new:nnn
{ advent }
{ missing file }
{ File~ #1~ doesn't~ exist~ \msg_line_context:}


% Caching

\prop_new:N \g__aoc_answers_prop

\file_get:nnNT {advent_answers.txt} {} \l__aoc_tmpa_tl
{
  \tl_use:N \l__aoc_tmpa_tl
}

\prop_gremove:Nn \g__aoc_answers_prop {Template}

\AddToHook {enddocument} 
{
  \iow_open:Nn \l__aoc_tmpa_iow {advent_answers.txt}
  \prop_map_inline:Nn \g__aoc_answers_prop
  {
    \iow_now:Nn \l__aoc_tmpa_iow
    {
      \prop_gput:Nnn \g__aoc_answers_prop {#1} {#2}
    }
  }
  \iow_close:N \l__aoc_tmpa_iow
}

% Clear the outputs
\cs_new_protected_nopar:Npn \aoc_clear_outputs:
{
  \tl_gclear:N \g__aoc_output_tl
  \int_gzero:N \g__aoc_output_int
  \seq_gclear:N \g__aoc_output_seq
  \bool_gset_false:N \g__aoc_output_bool
  \prop_gclear:N \g__aoc_output_prop
}


\tl_const:Nn \c__aoc_tl_tl {tl}
\tl_const:Nn \c__aoc_int_tl {int}
\tl_const:Nn \c__aoc_seq_tl {seq}
\tl_const:Nn \c__aoc_bool_tl {bool}
\tl_const:Nn \c__aoc_prop_tl {prop}
\bool_new:N \g__aoc_regenerate_next_bool

\cs_new_protected_nopar:Npn \aoc_check_answer:nnn #1#2#3
{
  \group_begin:
  \tl_set:Nn \l__aoc_tmpb_tl {#1}
  \bool_if:nTF
  {
    !\g__aoc_regenerate_next_bool
    &&
    \prop_if_in_p:Nn \g__aoc_answers_prop {#2}
  }
  {
    \prop_get:NnN \g__aoc_answers_prop {#2} \l__aoc_tmpa_tl
    \tl_case:Nn \l__aoc_tmpb_tl
    {
      \c__aoc_tl_tl {\tl_gset_eq:NN \g__aoc_output_tl \l__aoc_tmpa_tl}
      \c__aoc_int_tl {\int_gset:Nn \g__aoc_output_int {\tl_use:N \l__aoc_tmpa_tl}}
      \c__aoc_seq_tl {\exp_args:NNV\seq_gset_from_clist:Nn \g__aoc_output_seq \l__aoc_tmpa_tl}
    }
  }
  {
    \iow_log:n {Calculating~ solution~ for~ #2.}
    #3
    \tl_case:Nn \l__aoc_tmpb_tl
    {
      \c__aoc_tl_tl {\prop_gput:NnV \g__aoc_answers_prop {#2} \g__aoc_output_tl}
      \c__aoc_int_tl {\prop_gput:NnV \g__aoc_answers_prop {#2} \g__aoc_output_int}
      \c__aoc_seq_tl
      {
        \prop_gput:Nnx \g__aoc_answers_prop {#2} {\seq_use:Nn \g__aoc_output_seq {,}}
      }
    }
  }
  \bool_gset_false:N \g__aoc_regenerate_next_bool
  \group_end:
}
\cs_generate_variant:Nn \_aoc_check_answer:nnn {nVn,nxn}


% Day 01: Counting Calories
\cs_new_protected_nopar:Npn \aoc_max_calories:n #1
{
  \group_begin:
  % Try to open the file
  \ior_open:NnTF \l__aoc_tmpa_ior {#1 .txt}
  {
    % That was successful, so now initialise our variables.
    % We have an integer that carries the current count of calories and then a sequence that stores the current top three counts.
    \int_zero:N \l__aoc_tmpa_int
    \seq_clear:N \l__aoc_tmpa_seq
    \seq_put_right:Nn \l__aoc_tmpa_seq {0}
    \seq_put_right:Nn \l__aoc_tmpa_seq {0}
    \seq_put_right:Nn \l__aoc_tmpa_seq {0}

    \ior_map_inline:Nn \l__aoc_tmpa_ior
    {
      \tl_if_eq:nnTF {##1} {\par}
      {
        % Add our current count to the sequence
        \seq_put_right:NV \l__aoc_tmpa_seq \l__aoc_tmpa_int
        % Sort the sequence
        \seq_sort:Nn \l__aoc_tmpa_seq {
          \int_compare:nNnTF { ####1 } > { ####2 }
          { \sort_return_swapped: }
          { \sort_return_same: }
        }
        % Remove the lowest
        \seq_pop_left:NN \l__aoc_tmpa_seq \l__aoc_tmpa_tl
        % Zero the count
        \int_zero:N \l__aoc_tmpa_int
      }
      {
        \int_add:Nn \l__aoc_tmpa_int {##1}
      }
    }      
    % Add our current count to the sequence
    \seq_put_right:NV \l__aoc_tmpa_seq \l__aoc_tmpa_int
    % Sort the sequence
    \seq_sort:Nn \l__aoc_tmpa_seq {
      \int_compare:nNnTF { ##1 } > { ##2 }
      { \sort_return_swapped: }
      { \sort_return_same: }
    }
    % Remove the lowest
    \seq_pop_left:NN \l__aoc_tmpa_seq \l__aoc_tmpa_tl
    \seq_gset_eq:NN \g__aoc_output_seq \l__aoc_tmpa_seq
    \ior_close:N \l__aoc_tmpa_ior
  }
  {
    % If there was a problem with the file
    \msg_warning:nnn { advent } { missing file } {#1}
  }
  \group_end:
}

\DeclareDocumentCommand \MaxCalories { s m }
{
  \aoc_check_answer:nnn {seq} {Calories}
  {
    \aoc_max_calories:n {#2}
  }

  \int_zero:N \l__aoc_tmpa_int
  \IfBooleanTF {#1}
  {
    \int_set:Nn \l__aoc_tmpa_int
    {
      \seq_item:Nn \g__aoc_output_seq {3}
    }
  }
  {
    \int_add:Nn \l__aoc_tmpa_int
    {
      \seq_item:Nn \g__aoc_output_seq {1}
    }
    \int_add:Nn \l__aoc_tmpa_int
    {
      \seq_item:Nn \g__aoc_output_seq {2}
    }
    \int_add:Nn \l__aoc_tmpa_int
    {
      \seq_item:Nn \g__aoc_output_seq {3}
    }
  }
  \int_use:N \l__aoc_tmpa_int
  \int_zero:N \l__aoc_tmpa_int
  \aoc_clear_outputs:
}

% Day 02: Rock Paper Scissors

% First version
\cs_new_protected_nopar:Npn \aoc_rps_score_a:n #1
{
  \group_begin:
  % Try to open the file
  \ior_open:NnTF \l__aoc_tmpa_ior {#1 .txt}
  {
    \int_zero:N \l__aoc_tmpc_int
    \ior_map_inline:Nn \l__aoc_tmpa_ior
    {
      \tl_set:Nn \l__aoc_tmpa_tl {##1}
      \tl_set:Nx \l__aoc_tmpb_tl {\tl_item:Nn \l__aoc_tmpa_tl {2}}
      \tl_set:Nx \l__aoc_tmpa_tl {\tl_item:Nn \l__aoc_tmpa_tl {1}}
      \int_set:Nn \l__aoc_tmpa_int {\exp_args:Nc `{\l__aoc_tmpa_tl} - 65}
      \int_set:Nn \l__aoc_tmpb_int {\exp_args:Nc `{\l__aoc_tmpb_tl} - 88}

      % Add the score for the choice of RPS
      \int_add:Nn \l__aoc_tmpc_int {\l__aoc_tmpb_int + 1}
      % Add the score for the round
      \int_set:Nn \l__aoc_tmpa_int { \int_mod:nn {\l__aoc_tmpb_int - \l__aoc_tmpa_int + 4} {3} }
      \int_add:Nn \l__aoc_tmpc_int {3 * \l__aoc_tmpa_int}
      
    }
    \int_gset_eq:NN \g__aoc_output_int \l__aoc_tmpc_int
    \ior_close:N \l__aoc_tmpa_ior
  }
  {
    % If there was a problem with the file
    \msg_warning:nnn { advent } { missing file } {#1}
  }
  \group_end:
}

% Second version
\cs_new_protected_nopar:Npn \aoc_rps_score_b:n #1
{
  \group_begin:
  % Try to open the file
  \ior_open:NnTF \l__aoc_tmpa_ior {#1 .txt}
  {
    \int_zero:N \l__aoc_tmpc_int
    \ior_map_inline:Nn \l__aoc_tmpa_ior
    {
      \tl_set:Nn \l__aoc_tmpa_tl {##1}
      \tl_set:Nx \l__aoc_tmpb_tl {\tl_item:Nn \l__aoc_tmpa_tl {2}}
      \tl_set:Nx \l__aoc_tmpa_tl {\tl_item:Nn \l__aoc_tmpa_tl {1}}
      \int_set:Nn \l__aoc_tmpa_int {\exp_args:Nc `{\l__aoc_tmpa_tl} - 65}
      \int_set:Nn \l__aoc_tmpb_int {\exp_args:Nc `{\l__aoc_tmpb_tl} - 88}

      % Add the score for the choice of outcome
      \int_add:Nn \l__aoc_tmpc_int {3*(\l__aoc_tmpb_int )}
      % Figure out the play
      \int_set:Nn \l__aoc_tmpb_int { \int_mod:nn { \l__aoc_tmpa_int + \l__aoc_tmpb_int + 2 }{3} }
      \int_add:Nn \l__aoc_tmpc_int { \l__aoc_tmpb_int + 1}
      
    }
    \int_gset_eq:NN \g__aoc_output_int \l__aoc_tmpc_int
    \ior_close:N \l__aoc_tmpa_ior
  }
  {
    % If there was a problem with the file
    \msg_warning:nnn { advent } { missing file } {#1}
  }
  \group_end:
}

\DeclareDocumentCommand \RPSScore {s m}
{
  \IfBooleanTF {#1}
  {
    \aoc_check_answer:nnn {int} {RPSScoreA}
    {
      \aoc_rps_score_a:n {#2}
    }
  }
  {
    \aoc_check_answer:nnn {int} {RPSScoreB}
    {
      \aoc_rps_score_b:n {#2}
    }
  }
  \int_use:N \g__aoc_output_int
  \aoc_clear_outputs:
}

% Template

\cs_new_protected_nopar:Npn \aoc_template:n #1
{
  \group_begin:
  % Try to open the file
  \ior_open:NnTF \l__aoc_tmpa_ior {#1 .txt}
  {
    \ior_map_inline:Nn \l__aoc_tmpa_ior
    {
    }      
    \ior_close:N \l__aoc_tmpa_ior
  }
  {
    % If there was a problem with the file
    \msg_warning:nnn { advent } { missing file } {#1}
  }
  \group_end:
}

\NewDocumentCommand \Template {m}
{
  \bool_gset_true:N \g__aoc_regenerate_next_bool
  \aoc_check_answer:nnn {tl} {Template}
  {
    %
  }
  \tl_use:N \g__aoc_output_tl
  \aoc_clear_outputs:
}


\ExplSyntaxOff


\title{Advent of Code 2022: \LaTeX3 Solutions}
\author{Andrew Stacey \\ \url{loopspace@mathforge.org}}
\date{2022-12-01}

\renewcommand\thesubsection{Day \arabic{subsection}:}

\begin{document}

\maketitle

\section{Introduction}

This is my attempt to solve the Advent of Code 2022 problems in \LaTeX3.

Following from last year, I'm abandoning the \Verb!dtx! file format and using just a normal \LaTeX\ file and will input a \LaTeX3 file that actually contains the code.
From the outset I'm also using the caching system that I developed during last year's calendar.


\section{Solutions}

\subsection{Calories}

I'm not a fan of optimisation for its own sake, so for the second part of the challenge then I used a scratch sequence which I prepopulated with three zeros.
After working out each elf's quota, I added it to the sequence, then sorted the sequence, and removed the smallest value.
This is decidedly not optimal, but it was fast enough.

\begin{enumerate}
\item The largest amount of calories carried by a single elf is: \MaxCalories*{Day01}
\item The amount of calories carried by the three elves carrying the most is: \MaxCalories{Day01}
\end{enumerate}

\subsection{Rock, Paper, Scissors}

This one was a bit of modular arithmetic -- convert the letters to numbers \(0\) to \(2\), then use modular arithmetic to figure out who won.
If \(x\) is the opponent's play and \(y\) the elf's strategy, then the winning score is given by:
%
\[
  z = y - x \mod 3
  \]
%
where \(z = 0\) is a draw, \(-1\) a loss, and \(1\) a win.
To convert it to points, we need \(3 \times z\) for the game score and \(y + 1\) for the points for the play.

If given the desired outcome, we can recover \(y\) via \(x + z \mod 3\).

The only catch is that \LaTeX3 uses signed modular arithmetic, so \(-2 \mod 3\) is \(-2\).
To guard against this, we add \(3\) from time to time.

\begin{enumerate}
\item The score for the simple strategy is: \RPSScore*{Day02}
\item The score for the trickier strategy is: \RPSScore{Day02}
\end{enumerate}

\subsection{Rucksacks}

Fairly straightforward, this one.
I toyed with being a bit creative on the ways to figure out the overlaps between the token lists, but in the end \Verb!\tl_if_in! was too straightforward to ignore.
I didn't know about \Verb!\tl_range:nnn! before this, which is a nice thing to add to my repertoire.

\begin{enumerate}
\item The sum of the priorities of the overlapping items in the rucksacks is: \Rucksack*{Day03}
\item The sum of the priorities of the badges in the rucksacks is: \Rucksack{Day03}
\end{enumerate}

\subsection{Camp Cleanup}

The nice thing about this one is that parsing the input is really easy with \TeX\ due to the more complicated parameter specification.
I defined an auxiliary function that had a parameter pattern of \Verb!#1-#2,#3-#4! and fed it each line.

For the question as to whether the intervals overlap, it's easier to determine the non-overlaps than the overlaps: \([a,b]\) and \([c,d]\) don't overlap if \(b < c\) or \(d < a\).

\begin{enumerate}
\item The number of completely redundant cleanups is: \CampCleanup*{Day04}
\item The number of cleanups with overlaps is: \CampCleanup{Day04}
\end{enumerate}

\subsection{Supply Stacks}

This challenge presents two problems.
The first is an irritation with how the input is presented: the initial states of the crates is described as a set of \emph{vertical} lists.
This means that each line corresponds to a level rather than to a stack.
So the data has to be read with full positional information intact, whereas \TeX\ has a nasty habit of swallowing spaces.
So I needed to use \Verb!\ior_str_map_inline! for this one to read in the input as type \Verb!str! rather than \Verb!tl!.

The second is more about implementation.
The \LaTeX3 team have provided a decent set of data types, with property lists, sequences, token lists, integers, and so on.
However, \TeX's underlying system -- as a macro expansion language -- makes it tricky to define derived data types.
It's hard to have a sequence of sequences, for example.
Or even a sequence of token lists where the individual token lists can be acted upon whilst staying as part of the sequence.
It relates to the fact that \TeX\ more naturally works as pass-by-value rather than pass-by-reference.
While the latter is possible (it's the \Verb!N! type in \LaTeX3), then it isn't really full pass-by-reference because we're passing in the \emph{name} of the variable, which could then conflict with names used within the function.
Tied up with this is that anonymous variables don't really exist.

So what I'd really like here is a sequence of token lists where I could manipulate each token list while maintaining its position in the sequence.
Rather than try to figure out how to do that properly, I went for the easier option of using dynamic names and building the index into the naming convention.
So rather than having an actual sequence, I had macros of the form \Verb!\l__aoc_stack_1_tl!, only to get the \Verb!1! in there I had to make good use of the \Verb!c! argument.

\begin{enumerate}
\item With the CrateMaster9000, the top crates on the stacks were: \SupplyStacks*{Day05}
\item With the CrateMaster9001, the top crates on the stacks were: \SupplyStacks{Day05}
\end{enumerate}

\section{Tuning Troubles}

I spent too long with this one trying to get \TeX\ to read a file character by character.
What I wanted was to read the characters and pass them to an auxiliary macro that tests them.
I got as far as reading in the individual characters, but it turns out that the end-of-file character is \Verb!\outer! and can't be passed to a macro, so I would have to use a \Verb!\let! to get the next token and work with it.
Although there are the \Verb!\peek_after! macros, I couldn't figure out the right way to do it.
So in the end I gave up and read the whole file in to a token list and then stepped through that.
After that, it was relatively straightforward -- except that for the ``easy'' version then I hard-coded the tests and then had to reimplement them in loops for the second version.


\begin{enumerate}
\item The first start-of-packet marker is at \TuningTrouble*{Day06}
\item The first start-of-message marker is at \TuningTrouble{Day06}
\end{enumerate}

\section{Device Left on Space}

This one \emph{just worked}.
Well, once I'd sorted out the various methods for \Verb!prop!s, that is.
I used a sequence to keep track of the current directory, then as I descended into a directory, I'd zero a counter, add up the file sizes in that directory using that counter, and then store the sum in a prop indexed by the full directory paths.
When exiting a directory, I'd take the size for that directory and add it to whatever had already been calculated for its parent.
This ensured that subdirectory sizes propagated up to their parents.
After that, it was just a matter of sifting through the \Verb!prop! to find the answers sought.

\begin{enumerate}
\item The size of the small directories is \DeviceSpace*{Day07}
\item The size of the directory to delete is \DeviceSpace{Day07}
\end{enumerate}

\end{document}
